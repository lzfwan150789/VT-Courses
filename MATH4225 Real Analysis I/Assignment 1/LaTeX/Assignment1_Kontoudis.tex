
\documentclass[12pt]{article}
 
\usepackage[margin=1in]{geometry} 
\usepackage{amsmath,amsthm,amssymb}
\usepackage{pifont}
\usepackage{graphics}
\usepackage{graphicx}
\usepackage{array}
 
\newcommand{\N}{\mathbb{N}}
\newcommand{\Z}{\mathbb{Z}}
 
\newenvironment{theorem}[2][Theorem]{\begin{trivlist}
\item[\hskip \labelsep {\bfseries #1}\hskip \labelsep {\bfseries #2.}]}{\end{trivlist}}
\newenvironment{lemma}[2][Lemma]{\begin{trivlist}
\item[\hskip \labelsep {\bfseries #1}\hskip \labelsep {\bfseries #2.}]}{\end{trivlist}}
\newenvironment{exercise}[2][Exercise]{\begin{trivlist}
\item[\hskip \labelsep {\bfseries #1}\hskip \labelsep {\bfseries #2.}]}{\end{trivlist}}
\newenvironment{problem}[2][Problem]{\begin{trivlist}
\item[\hskip \labelsep {\bfseries #1}\hskip \labelsep {\bfseries #2.}]}{\end{trivlist}}
\newenvironment{question}[2][Question]{\begin{trivlist}
\item[\hskip \labelsep {\bfseries #1}\hskip \labelsep {\bfseries #2.}]}{\end{trivlist}}
\newenvironment{corollary}[2][Corollary]{\begin{trivlist}
\item[\hskip \labelsep {\bfseries #1}\hskip \labelsep {\bfseries #2.}]}{\end{trivlist}}

\newenvironment{solution}{\begin{proof}[Solution]}{\end{proof}}
 
\begin{document}
 
% --------------------------------------------------------------
%                         Start here
% --------------------------------------------------------------
 
\title{Problem Set 1}
\author{Georgios Kontoudis \textbullet{} gpkont@vt.edu\\ 
MATH4225 Elementary Real Analysis\\
Instructor: Dr. Jacob Fillman} 
\date{Fall 2017}
 
\maketitle
\begin{exercise}{1} %theorem, exercise, problem, or question 
Negate the statement: ``For every $\epsilon >0$, there exists $\delta >0$ such that $|f(x)-f(y)|< \epsilon$ whenever $|x-y|< \delta$. ''
\end{exercise}
\begin{solution}
``There exists $\epsilon >0$, such that for every $\delta >0$, $|f(x)-f(y)|< \epsilon$ is not true whenever $|x-y|< \delta$.'' 
\end{solution}

\begin{exercise}{2 (a)} %theorem, exercise, problem, or question 
Show that there is no rational number r such that 
\begin{equation}\label{eq2a0}
r^2 = 32.
\end{equation}
\end{exercise}
\begin{solution}
Suppose on the contrary that, $r\in \mathbb{Q}$ and $r^2=32$. Then, from the definition of rational numbers we get $r=\frac{a}{b}$, with $a,b \in \mathbb{Z}$, $b\neq0$, and $a,b$ share no common factors. Equation~\ref{eq2a0} implies
\begin{equation}\label{eq2a1}
a^2=32b^2 \Rightarrow a^2 = 2^4 b^2.
\end{equation}
This means that $a^2$ is even. Thus, $a$ is even because if it was odd $a^2$ would be odd. Therefore, we have $a=2d, d\in \mathbb{Z}$ and Equation~\ref{eq2a1} yields
\begin{equation}
2^2d^2=2^4b^2 \Rightarrow b^2=2^{-2}d^2.
\end{equation}
For the same reason as above, $b$ is even and since $a,b$ cannot share common factors they cannot be both even. Thus, r is not rational $r \notin \mathbb{Q}$.
\end{solution}

\begin{exercise}{2 (b)} %theorem, exercise, problem, or question 
Show that the set $E=\{ r\in \mathbb{Q}: r^2<32 \}$ has no least upper bound.
\end{exercise}
\begin{solution}
The set $E$ is nonempty and is bounded above by any $x> \sqrt{32}\notin \mathbb{Q} $. We consider the set $F=\{ x\in \mathbb{Q}:x>0, x^2>32 \}$ as the set of the upper bounds of $E$ in $\mathbb{Q}$. We claim that $F$ has no least element, which means that for every $p\in F$ there exists a $q\in F$ such that $q<p$. We associate every rational $p>0$, by using the secant method, with 
\begin{equation}\label{eq_2b1}
q=p-\frac{f(p)(32-p)}{f(32)-f(p)}=p-\frac{p^2-2^5}{p+2^5}=\frac{2^5p+2^5}{2^5+p}.
\end{equation}
Next, we employ the secant method's properties, as $q $ has the structure
\begin{equation}\label{eq_2b2}
q=\frac{\alpha p+\alpha}{\alpha + p}.
\end{equation}
If $p^2-2^5<0$, then from Equation~\ref{eq_2b1} we get that $q>p$, and Equation~\ref{eq_2b2} yields $q^2<2^5$, while for $p^2-2^5>0$ Equation~\ref{eq_2b1} returns $q<p $ and Equation~\ref{eq_2b2} yields $q^2>2^5$. Thus, $E $ has no supremum (least upper bound) in $\mathbb{Q}$.
\end{solution}

\begin{exercise}{3} %theorem, exercise, problem, or question 
For $n\in \mathbb{N}$ with $n\geq 2$, $\sqrt{n}$ is irrational if and only if its prime factorization \underline{contains an odd power of a prime}.
\end{exercise}
\begin{solution}
For the sake of argument let $\sqrt{n}$ be in $\mathbb{Q}$, then $\sqrt{n}=\frac{a}{b}$ where $a,b \in \mathbb{Z}$. If we square both sides then we get $n=\frac{a^2}{b^2} \Rightarrow b^2n=a^2$. We employ the fundamental theorem of arithmetic and we obtain $a^2=p_1^{2 \alpha_1}p_2^{2 \alpha_2} \hdots p_k^{2 \alpha_k}$, where every unique element has even power. Similarly, $b^2$ has even powers. Thus, $n=q_1^{2 \beta_1}q_2^{2 \beta_2} \hdots q_k^{2 \beta_k}$, which means that $n=l^2$ where $l=q_1^{\beta_1}q_2^{\beta_2} \hdots q_k^{\beta_k}$. Therefore, if the prime factorization of $\sqrt{n}$ has only even power of primes, then it is rational.
\end{solution}

\begin{exercise}{4} %theorem, exercise, problem, or question 
If $r \neq 0$ is a rational number and $x$ is irrational, prove that $r+x$ and $rx$ are irrational.
\end{exercise}
\begin{solution}
Assume, to contrary that $r+x$ and $rx$ are rational, then we get $r+x=r+x \Rightarrow x=r+x-r$ which means that x is rational. Similarly, $rx=rx \Rightarrow x=\frac{rx}{r}$ which means that x is rational. Thus, x is irrational so the addition $r+x $ and the product $rx$ are irrational too.
\end{solution}

\begin{exercise}{5} %theorem, exercise, problem, or question 
Let
\begin{equation}
E=\{ 11+(-1)^n \Big(3-\frac{5}{n^3} \Big):n \in \mathbb{N} \}.
\end{equation}
Identify inf $E$ and sup $E$.
\end{exercise}
\begin{solution}
The set $E$ for even numbers yields
\begin{equation}
E_{even}=\{ 14-\frac{5}{n^3}:n=2k, k \in \mathbb{N} \},
\end{equation}
while for odd numbers we get
\begin{equation}
E_{odd}=\{ 8+\frac{5}{n^3} :n=2k+1, k=0, k \in \mathbb{N} \}.
\end{equation}
Let the candidate for supremum be 14. We first need to check if 14 is an upper bound, which means that for all $e \in E$, $e \leq 14$. If $e \in E$, we get for the odd numbers
\begin{equation}
8+\frac{5}{n^3} \leq 14 \Rightarrow \frac{5}{6} \leq n^3,
\end{equation}
which is true for all odd numbers $n=2k+1, k=0, k \in \mathbb{N}$. Similarly, if $e \in E$, we get for the even numbers
\begin{equation}
14-\frac{5}{n^3} \leq 14 \Rightarrow \frac{5}{n^3} \geq 0,
\end{equation}
which is also true for all even numbers $n=2k, k \in \mathbb{N}$. Thus, 14 is an upper bound for all $e \in E$.

Next, let the candidate for infimum be 8. We need to check if 8 is a lower bound, which means that for all $e \in E$, $e \geq 8$. If $e \in E$, we get for the odd numbers
\begin{equation}
8+\frac{5}{n^3} \geq 8 \Rightarrow \frac{5}{n^3} \geq 0,
\end{equation}
which is true for all odd numbers $n=2k+1, k=0, k \in \mathbb{N}$. Similarly, if $e \in E$, we get for the even numbers
\begin{equation}
14-\frac{5}{n^3} \geq 8 \Rightarrow \frac{5}{6} \leq n^3,
\end{equation}
which is also true for all even numbers $n=2k, k \in \mathbb{N}$. Thus, 8 is a lower bound for all $e \in E$.

Before we continue with the supremum and infimum, we observe that $3-\frac{5}{n^3}>0$ for all $n \in \mathbb{N}$ except $n=1$ which yields $e=13$, that is neither a lower bound nor an upper bound. Assume, to the contrary that $e=13$ is an upper bound, which means that for all $e \in E$, $e \leq 13$. Then, if $e \in E$, we get for the even numbers, $(14-\frac{5}{n^3}) \leq 13 \Rightarrow \frac{5}{n^3} \geq 1$ which is false for all even numbers $n=2k, k \in \mathbb{N}$, and thus 13 is not an upper bound. Assume, to the contrary that $e=13$ is a lower bound, which means that for all $e \in E$, $e \geq 13$. Then, if $e \in E$, we get for the odd numbers, $(8+\frac{5}{n^3}) \geq 13 \Rightarrow n^3 \leq 1$ which is false for all odd numbers $n=2k+1, k \in \mathbb{N}$, and thus 13 is not a lower bound. 

Then, we will show that sup $E$=14 and inf $E$=8. We observe that $\frac{5}{n^3}<1$ for all $n= \mathbb{N}, n\neq 1$, which means that the set $E$ converges to 14 for all even values $n=2k, k \in \mathbb{N}$ and to 8 for all odd values $n=2k+1, k\neq 0, k \in \mathbb{N}$. So we wil study even numbers for the supremum and odd numbers for the infimum.

Since $e=14$ is an upper bound of $E$, $(14-\frac{5}{n^3}) \leq 14$ for all $n=2k, k \in \mathbb{N}$. Assume, to the contrary that $e<14$. Thus, $14-e>0$ and by the Archimedean Property, there exists an $n^3$ for all $n=2k, k \in \mathbb{N}$ such that $n^3(14-e)>5$ that yields
\begin{equation}
e<14-\frac{5}{n^3} \in E, 
\end{equation}
which contradicts that $e$ is an upper bound. Therefore, sup $E$=14. 

For the infimum, since $e=8$ is a lower bound of $E$, $(8+\frac{5}{n^3}) \geq 8$ for all $n=2k, k \in \mathbb{N}$. Assume, to the contrary that $e>8$. Thus, $e-8>0$ and by the Archimedean Property, there exists an $n^3$ for all $n=2k+1, k \in \mathbb{N}$ such that $n^3(e-8)>5$ that yields
\begin{equation}
e>8+\frac{5}{n^3} \in E, 
\end{equation} 
which contradicts that $e$ is a lower bound. Therefore, inf $E$=8.
 
\end{solution}

\begin{exercise}{6 (a)} %theorem, exercise, problem, or question 
Many math texts adopt the conventions sup $\emptyset$ = $- \infty$ and inf $\emptyset$ = $+ \infty$. Discuss why this is reasonable convention.
\end{exercise}
\begin{solution}
Any real number is an upper bound of the empty set, so $- \infty$ can be the least, and thus the supremum. Similarly, the infimum of an empty set is $+ \infty$, since any real number is a lower bound of the empty set, and thus $+ \infty$ would be the greatest.
\end{solution}

\begin{exercise}{6 (b)} %theorem, exercise, problem, or question 
Adopting this convention, show that, for $E \subset \mathbb{R}$, one has inf $E$ $\leq$ sup $E$ \textit{if and only if} $E \neq \emptyset$.
\end{exercise}
\begin{solution}
For non-empty sets we inherit the definition of upper bound and lower bound. We call $u \in \mathbb{R}$ an upper bound for $E$ if $x\leq u$ for all $x\in E$. We call $u \in \mathbb{R}$ a lower bound for $E$ if $x\geq u$ for all $x\in E$. For a nonempty set $E\neq \emptyset$, that has two or more elements we get inf $E$ $\leq x \leq y \leq $ sup $E$. Note that for a single element in a set we have inf $E$ $=$ sup $E$. 
\end{solution}

\end{document}


